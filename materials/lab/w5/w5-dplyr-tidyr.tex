\documentclass[11pt,]{article}
\usepackage{lmodern}
\usepackage{amssymb,amsmath}
\usepackage{ifxetex,ifluatex}
\usepackage{fixltx2e} % provides \textsubscript
\ifnum 0\ifxetex 1\fi\ifluatex 1\fi=0 % if pdftex
  \usepackage[T1]{fontenc}
  \usepackage[utf8]{inputenc}
\else % if luatex or xelatex
  \ifxetex
    \usepackage{mathspec}
  \else
    \usepackage{fontspec}
  \fi
  \defaultfontfeatures{Ligatures=TeX,Scale=MatchLowercase}
\fi
% use upquote if available, for straight quotes in verbatim environments
\IfFileExists{upquote.sty}{\usepackage{upquote}}{}
% use microtype if available
\IfFileExists{microtype.sty}{%
\usepackage{microtype}
\UseMicrotypeSet[protrusion]{basicmath} % disable protrusion for tt fonts
}{}
\usepackage[margin=1cm]{geometry}
\usepackage{hyperref}
\PassOptionsToPackage{usenames,dvipsnames}{color} % color is loaded by hyperref
\hypersetup{unicode=true,
            pdftitle={More dplyr and tidyr},
            pdfauthor={Haohan Chen},
            colorlinks=true,
            linkcolor=blue,
            citecolor=Blue,
            urlcolor=Blue,
            breaklinks=true}
\urlstyle{same}  % don't use monospace font for urls
\usepackage{color}
\usepackage{fancyvrb}
\newcommand{\VerbBar}{|}
\newcommand{\VERB}{\Verb[commandchars=\\\{\}]}
\DefineVerbatimEnvironment{Highlighting}{Verbatim}{commandchars=\\\{\}}
% Add ',fontsize=\small' for more characters per line
\usepackage{framed}
\definecolor{shadecolor}{RGB}{248,248,248}
\newenvironment{Shaded}{\begin{snugshade}}{\end{snugshade}}
\newcommand{\KeywordTok}[1]{\textcolor[rgb]{0.13,0.29,0.53}{\textbf{#1}}}
\newcommand{\DataTypeTok}[1]{\textcolor[rgb]{0.13,0.29,0.53}{#1}}
\newcommand{\DecValTok}[1]{\textcolor[rgb]{0.00,0.00,0.81}{#1}}
\newcommand{\BaseNTok}[1]{\textcolor[rgb]{0.00,0.00,0.81}{#1}}
\newcommand{\FloatTok}[1]{\textcolor[rgb]{0.00,0.00,0.81}{#1}}
\newcommand{\ConstantTok}[1]{\textcolor[rgb]{0.00,0.00,0.00}{#1}}
\newcommand{\CharTok}[1]{\textcolor[rgb]{0.31,0.60,0.02}{#1}}
\newcommand{\SpecialCharTok}[1]{\textcolor[rgb]{0.00,0.00,0.00}{#1}}
\newcommand{\StringTok}[1]{\textcolor[rgb]{0.31,0.60,0.02}{#1}}
\newcommand{\VerbatimStringTok}[1]{\textcolor[rgb]{0.31,0.60,0.02}{#1}}
\newcommand{\SpecialStringTok}[1]{\textcolor[rgb]{0.31,0.60,0.02}{#1}}
\newcommand{\ImportTok}[1]{#1}
\newcommand{\CommentTok}[1]{\textcolor[rgb]{0.56,0.35,0.01}{\textit{#1}}}
\newcommand{\DocumentationTok}[1]{\textcolor[rgb]{0.56,0.35,0.01}{\textbf{\textit{#1}}}}
\newcommand{\AnnotationTok}[1]{\textcolor[rgb]{0.56,0.35,0.01}{\textbf{\textit{#1}}}}
\newcommand{\CommentVarTok}[1]{\textcolor[rgb]{0.56,0.35,0.01}{\textbf{\textit{#1}}}}
\newcommand{\OtherTok}[1]{\textcolor[rgb]{0.56,0.35,0.01}{#1}}
\newcommand{\FunctionTok}[1]{\textcolor[rgb]{0.00,0.00,0.00}{#1}}
\newcommand{\VariableTok}[1]{\textcolor[rgb]{0.00,0.00,0.00}{#1}}
\newcommand{\ControlFlowTok}[1]{\textcolor[rgb]{0.13,0.29,0.53}{\textbf{#1}}}
\newcommand{\OperatorTok}[1]{\textcolor[rgb]{0.81,0.36,0.00}{\textbf{#1}}}
\newcommand{\BuiltInTok}[1]{#1}
\newcommand{\ExtensionTok}[1]{#1}
\newcommand{\PreprocessorTok}[1]{\textcolor[rgb]{0.56,0.35,0.01}{\textit{#1}}}
\newcommand{\AttributeTok}[1]{\textcolor[rgb]{0.77,0.63,0.00}{#1}}
\newcommand{\RegionMarkerTok}[1]{#1}
\newcommand{\InformationTok}[1]{\textcolor[rgb]{0.56,0.35,0.01}{\textbf{\textit{#1}}}}
\newcommand{\WarningTok}[1]{\textcolor[rgb]{0.56,0.35,0.01}{\textbf{\textit{#1}}}}
\newcommand{\AlertTok}[1]{\textcolor[rgb]{0.94,0.16,0.16}{#1}}
\newcommand{\ErrorTok}[1]{\textcolor[rgb]{0.64,0.00,0.00}{\textbf{#1}}}
\newcommand{\NormalTok}[1]{#1}
\usepackage{graphicx,grffile}
\makeatletter
\def\maxwidth{\ifdim\Gin@nat@width>\linewidth\linewidth\else\Gin@nat@width\fi}
\def\maxheight{\ifdim\Gin@nat@height>\textheight\textheight\else\Gin@nat@height\fi}
\makeatother
% Scale images if necessary, so that they will not overflow the page
% margins by default, and it is still possible to overwrite the defaults
% using explicit options in \includegraphics[width, height, ...]{}
\setkeys{Gin}{width=\maxwidth,height=\maxheight,keepaspectratio}
\IfFileExists{parskip.sty}{%
\usepackage{parskip}
}{% else
\setlength{\parindent}{0pt}
\setlength{\parskip}{6pt plus 2pt minus 1pt}
}
\setlength{\emergencystretch}{3em}  % prevent overfull lines
\providecommand{\tightlist}{%
  \setlength{\itemsep}{0pt}\setlength{\parskip}{0pt}}
\setcounter{secnumdepth}{0}
% Redefines (sub)paragraphs to behave more like sections
\ifx\paragraph\undefined\else
\let\oldparagraph\paragraph
\renewcommand{\paragraph}[1]{\oldparagraph{#1}\mbox{}}
\fi
\ifx\subparagraph\undefined\else
\let\oldsubparagraph\subparagraph
\renewcommand{\subparagraph}[1]{\oldsubparagraph{#1}\mbox{}}
\fi

%%% Use protect on footnotes to avoid problems with footnotes in titles
\let\rmarkdownfootnote\footnote%
\def\footnote{\protect\rmarkdownfootnote}

%%% Change title format to be more compact
\usepackage{titling}

% Create subtitle command for use in maketitle
\newcommand{\subtitle}[1]{
  \posttitle{
    \begin{center}\large#1\end{center}
    }
}

\setlength{\droptitle}{-2em}

  \title{More \texttt{dplyr} and \texttt{tidyr}}
    \pretitle{\vspace{\droptitle}\centering\huge}
  \posttitle{\par}
    \author{Haohan Chen\footnote{Political Science Department, Duke University.
  \href{mailto:haohan.chen@duke.edu}{\nolinkurl{haohan.chen@duke.edu}}}}
    \preauthor{\centering\large\emph}
  \postauthor{\par}
      \predate{\centering\large\emph}
  \postdate{\par}
    \date{October 5, 2018}


\begin{document}
\maketitle

\section{Download an example dataset}\label{download-an-example-dataset}

\begin{Shaded}
\begin{Highlighting}[]
\CommentTok{# Just to show how the data is downloaded, in case you are curious.}
\CommentTok{# Data downloade from ICPSR: https://www.icpsr.umich.edu/icpsrweb/ICPSR/}

\CommentTok{# if (!"psData" %in% installed.packages()[, 1])\{}
\CommentTok{#   install.packages("psData")}
\CommentTok{# \}}
\CommentTok{# }
\CommentTok{# library(psData)}
\CommentTok{# polity <- PolityGet()}
\CommentTok{# save(polity, file = "polity.Rdata")}
\end{Highlighting}
\end{Shaded}

\section{Load the data and libraries}\label{load-the-data-and-libraries}

\begin{Shaded}
\begin{Highlighting}[]
\CommentTok{# Libraries}
\KeywordTok{library}\NormalTok{(dplyr)}
\KeywordTok{library}\NormalTok{(tidyr)}

\CommentTok{# Data}
\KeywordTok{load}\NormalTok{(}\StringTok{"polity.Rdata"}\NormalTok{)}
\NormalTok{polity <-}\StringTok{ }\KeywordTok{as_tibble}\NormalTok{(polity)}
\end{Highlighting}
\end{Shaded}

\section{Review: What you have
learned}\label{review-what-you-have-learned}

\begin{itemize}
\tightlist
\item
  \texttt{select}, \texttt{filter}, \texttt{arrange}, \texttt{mutate}.
  What do they do?
\item
  Exercise: Get the variables \texttt{country} and \texttt{polity} of
  the ``Uganda'' from 1990-2000, sort it form largest to smallest.
\end{itemize}

Note: Note that POLITY score captures political regime authority
spectrum on a 21-pont scale ranging from -10 (hereditary monarchy) to
+10 (consolidated democracy). The Polity scores can also be converted
into regime categories in a suggested three part categorization of
``autocracies'' (-10 to -6), ``anocracies'' (-5 to +5 and three special
values: -66, -77 and -88), and ``democracies'' (+6 to +10). Performance
score from 0 to 100. The highest score reflects the best situation.

\begin{Shaded}
\begin{Highlighting}[]
\NormalTok{polity }\OperatorTok\StringTok{ }\KeywordTok{filter}\NormalTok{(country }\OperatorTok{==}\StringTok{ "Brazil"}\NormalTok{) }\OperatorTok\StringTok{ }\KeywordTok{select}\NormalTok{(country, year, polity2) }\OperatorTok\StringTok{ }\KeywordTok{arrange}\NormalTok{(}\OperatorTok{-}\NormalTok{polity2)}
\end{Highlighting}
\end{Shaded}

\begin{verbatim}
## # A tibble: 192 x 3
##    country  year polity2
##    <chr>   <dbl>   <dbl>
##  1 Brazil   1988       8
##  2 Brazil   1989       8
##  3 Brazil   1990       8
##  4 Brazil   1991       8
##  5 Brazil   1992       8
##  6 Brazil   1993       8
##  7 Brazil   1994       8
##  8 Brazil   1995       8
##  9 Brazil   1996       8
## 10 Brazil   1997       8
## # ... with 182 more rows
\end{verbatim}

\section{New}\label{new}

\begin{itemize}
\tightlist
\item
  group\_by, summarise: Get the mean, maximum, minimum, median, standard
  deviation of \texttt{polity2} for each country from 1990-2000.
\item
  slice: Get the first 10 rows
\item
  reshape dataset: long \textless{}- wide: Get a wide-form dataset of
  2000-2005, Each year as a column
\end{itemize}

\begin{Shaded}
\begin{Highlighting}[]
\NormalTok{polity }\OperatorTok\StringTok{ }
\StringTok{  }\KeywordTok{filter}\NormalTok{(year }\OperatorTok\StringTok{ }\KeywordTok{c}\NormalTok{(}\DecValTok{2000}\OperatorTok{:}\DecValTok{2005}\NormalTok{)) }\OperatorTok
\StringTok{  }\KeywordTok{select}\NormalTok{(country, year, polity2) }\OperatorTok\StringTok{ }
\StringTok{  }\KeywordTok{spread}\NormalTok{(year, polity2)}
\end{Highlighting}
\end{Shaded}

\begin{verbatim}
## # A tibble: 163 x 7
##    country     `2000` `2001` `2002` `2003` `2004` `2005`
##    <chr>        <dbl>  <dbl>  <dbl>  <dbl>  <dbl>  <dbl>
##  1 Afghanistan     -7     NA     NA     NA     NA     NA
##  2 Albania          5      5      7      7      7      9
##  3 Algeria         -3     -3     -3     -3      2      2
##  4 Angola          -3     -3     -2     -2     -2     -2
##  5 Argentina        8      8      8      8      8      8
##  6 Armenia          5      5      5      5      5      5
##  7 Australia       10     10     10     10     10     10
##  8 Austria         10     10     10     10     10     10
##  9 Azerbaijan      -7     -7     -7     -7     -7     -7
## 10 Bahrain         -9     -8     -7     -7     -7     -7
## # ... with 153 more rows
\end{verbatim}

\section{More information}\label{more-information}

\begin{itemize}
\tightlist
\item
  \textbf{Must read:}
  \url{https://www.rstudio.com/wp-content/uploads/2015/02/data-wrangling-cheatsheet.pdf}
\item
  Some tutorials with examples:

  \begin{itemize}
  \tightlist
  \item
    \url{https://rpubs.com/bradleyboehmke/data_wrangling}
  \item
    \url{http://garrettgman.github.io/tidying/}
  \end{itemize}
\end{itemize}


\end{document}
